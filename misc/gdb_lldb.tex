\begin{tabular}{|l|l|}
    \hline
    \multicolumn{1}{|c|}{\textbf{GDB}}
    &
    \multicolumn{1}{c|}{\textbf{LLDB}}\\
    \hline
    
    \hline
    \multicolumn{2}{|c|}{\textbf{Correr procesos}}\\
    \hline
    % GDB
    \begin{tabular}{@{}l@{}}
        run <args>\\
        r <args>\\
    \end{tabular}
    & % LLDB
    \begin{tabular}{@{}l@{}}
        process launch <args>\\
        run <args>\\
        r <args>\\
    \end{tabular}\\
    \hline

    \hline
    \multicolumn{2}{|c|}{\textbf{Argumentos}}\\
    \hline
    % GDB
    \begin{tabular}{@{}l@{}}
        \textcolor{OliveGreen}{\%} gdb --args a.out 1 2 3\\
        set args 1 2 3\\
        \textcolor{gray}{// Mostrar args}\\
        show args\\
    \end{tabular}
    & % LLDB
    \begin{tabular}{@{}l@{}}
        \textcolor{OliveGreen}{\%} lldb -- a.out 1 2 3\\
        settings set target.run-args 1 2 3\\
        \textcolor{gray}{// Mostrar args}\\
        settings show target.run-args\\
    \end{tabular}\\
    \hline

    \hline
    \multicolumn{2}{|c|}{\textbf{Variables de entorno}}\\
    \hline
    % GDB
    \begin{tabular}{@{}l@{}}
        set env DEBUG=1\\
        unset env DEBUG\\
    \end{tabular}
    & % LLDB
    \begin{tabular}{@{}l@{}}
        settings set target.env-vars DEBUG=1\\
        set se target.env-vars DEBUG=1\\
        env DEBUG=1\\
        settings remove target.env-vars DEBUG\\
        set rem target.env-vars DEBUG\\
    \end{tabular}\\
    \hline

    \hline
    \multicolumn{2}{|c|}{\textbf{Avanzar}}\\
    \hline
    % GDB
    \begin{tabular}{@{}l@{}}
        \textcolor{gray}{// Step in}\\
        step\\
        s\\
        \textcolor{gray}{// Step over}\\
        next\\
        n\\
        \textcolor{gray}{// Step out}\\
        finish\\
        \textcolor{gray}{// Return}\\
        return <RETURN EXPRESSION>\\
    \end{tabular}
    & % LLDB
    \begin{tabular}{@{}l@{}}
        \textcolor{gray}{// Step in}\\
        thread step-in\\
        step\\
        s\\
        \textcolor{gray}{// Step over}\\
        thread step-over\\
        next\\
        n\\
        \textcolor{gray}{// Step out}\\
        thread step-out\\
        finish\\
        \textcolor{gray}{// Return}\\
        thread return <RETURN EXPRESSION>\\
    \end{tabular}\\
    \hline

    \hline
    \multicolumn{2}{|c|}{\textbf{Breakpoints}}\\
    \hline
    % GDB
    \begin{tabular}{@{}l@{}}
        break main\\
        b test.cpp:12\\
        \textcolor{gray}{// Set bp con condicional}\\
        break foo if n == 43\\
        \textcolor{gray}{// Listar bp}\\
        info break\\
        \textcolor{gray}{// Modificar o quitar la condicional del x-ésimo bp}\\
        cond x <CONDITION>\\
        \textcolor{gray}{// (Des)activar y eliminar bp}\\
        disable x\\
        enable x\\
        delete x\\
    \end{tabular}
    & % LLDB
    \begin{tabular}{@{}l@{}}
        breakpoint set --name main\\
        br s -f test.cpp -l 12\\
        b test.cpp:12 \textcolor{gray}{// b SÓLO SETEA!!}\\
        \textcolor{gray}{// Set bp con condicional}\\
        br s -n solve -c '(int)n == 43'\\
        \textcolor{gray}{// Listar bp}\\
        breakpoint list\\
        br l\\
        \textcolor{gray}{// Modificar o quitar la condicinal del x-ésimo bp}\\
        br m -c '<CONDITION>' x\\
        \textcolor{gray}{// (Des)activar y eliminar bp}\\
        breakpoint disable x\\
        br en x\\
        br delete x\\
    \end{tabular}\\
    \hline
\end{tabular}
